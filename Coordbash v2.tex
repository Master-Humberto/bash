\documentclass{article}
\usepackage{graphicx} % Required for inserting images
\usepackage{draftwatermark}
\usepackage{tikz}
\usepackage{amsmath}
\usepackage{emoji}
\SetWatermarkText{BASH}
\SetWatermarkScale{1}
\usepackage{fontspec}
\usetikzlibrary{calc}
\usepackage{cancel}
\setmainfont{Comic Sans MS}
\usepackage{lipsum}
\usepackage{emoji}
\usepackage[most]{tcolorbox}
\renewcommand*\contentsname{Summary}

\usepackage{hyperref}
\title{Coordbash}
\author{Master Humberto}
\date{Copia não comédia}

\begin{document}

\maketitle
\tableofcontents


\section{Geometria com contas, como começar ?}

Quando se pensa em geometria com contas, geralemente se pensa naquelas contas "sem pensar" que usam "setups" sem muita lógica pois, se a conta "bate", você pode ficar "despreocupado".
De certo modo, isso não deixa de ser verdade, mas, pensando num contexto lógico que você vá fazer uma olimpíada na qual você tem um tempo lmitado, gastar uns 5-10 minutinhos pensando na melhor forma de fazer o problema \textcolor{red}{pode te salvar 1 hora ou até mais} de contas futuramente.
Dessa forma, os módulos do curso começarão tratando de construção de "setups", evoluindo para os problemas que serão futuramente apresentados aqui.

\section{Como criar um setup para um problema de coordbash ?}

Os setup´s para  a realização do coordbash são relativamente simples. Você precisará definir onde ficarão os eixos x e y e para onde eles estão apontando. Geralmente, para essa escolha, é conveniente pensar em pontos na qual tem muitas retas concorrendo como o (0,0) ou então centros de círculos de modo a facilitar as contas. Em coordbash, temos que o "0" é tudo, pois ele facilita bastante as contas posteriormente. Existem infinitos setups que pode-se criar, mas os três mais comuns são usando a base de um triângulo o o outro vértice no eixo y, o centro de um círculo ou, então, diagonais que são perpendiculares.
\begin{center}
\begin{tikzpicture}
  % Draw x-axis
  \draw[thick, ->] (-5,0) -- (6,0) node[below] {$x$};
  
  % Draw y-axis
  \draw[thick, ->] (0,-1) -- (0,5) node[left] {$y$};
  
  % Draw the triangle and fill it with red
  \draw[red] (-3,0) -- (5,0) -- (0,4) -- cycle;
  
  % Label the vertices with their coordinates
  \node[below] at (-3,0) {$(-3,0)$};
  \node[below] at (5,0)  {$(5,0)$};
  \node[above] at (0,4)  {$(0,4)$};
  \node[below] at (0,-1) {Temos aqui um triângulo com a base no eixo x e o outro vértice pertencendo ao eixo y};
  % Add blue circles at the triangle vertices
  \filldraw[blue] (-3,0) circle (3pt);
  \filldraw[blue] (5,0) circle (3pt);
  \filldraw[blue] (0,4) circle (3pt);
\end{tikzpicture}
\end{center}

\section{A reta no plano euclidiano}

O que seria um reta no plano euclidiano ? Na escola aprende-se que todas as retas no plano são da forma $y = ax + b$, em que a é o coeficiente angular da reta e b é o coeficiente linear da reta. Mas o que cada um significa ? Bem, mas o que isso significa ? Temos que, basicamente, para cada x do gráfico, relaciona-se um ponto y que é calculado como $y = ax +b$. Bem intuitivo, não ? Tem-se que todas as retas do plano podem ser representadas variando os parâmetros a e b, variando o "a" você altera a inclinação da reta e alterando o b você coloca a reta para cima ou para baixo. Existem duas formas de achar os parâmetros a e b, a primeira é caso você tenha duas retas $y = a_1 x + b_1$ e $y = a_2 x + b_2$, então você faz um sistema de equações e descobre a e b, ou, então, você sabe que a reta passa por certo ponto e tem certo coeficiente angular ou linear, daí, você pode achar a outra informação que falta, ou seja, se tiver "a" se ahca "b", e vice-versa.

\subsection{Definição de "a"}
O "a" ou coeficiente angular da reta é basicamente igual à tangente do ângulo formado pela reta e o semieixo positivo das abcissas. Como você pode não entender isso de cara, vou desenhar para que você entenda.
\begin{center}

\begin{tikzpicture}
  % Draw x-axis
  \draw[thick, ->] (0,0) -- (5,0) node[below] {$x$};
  
  % Draw y-axis
  \draw[thick, ->] (0,0) -- (0,3) node[left] {$y$};

    \draw[dot]
  
  % Define the angle (in degrees)
  \def\angle{45}
  
  % Calculate the coordinates of the line's endpoint
  \pgfmathsetmacro\xcoord{cos(\angle)}
  \pgfmathsetmacro\ycoord{sin(\angle)}
  
  % Draw the line
  \draw[red, thick] (0,0) -- (5*\xcoord,5*\ycoord) node[midway, above] {};
  
  % Draw the arc
     \draw[<-, blue, thick] (1,0) arc (0:\angle:1) node[right, xshift=2mm, yshift=-1mm] {$\theta$};
  \fill[black] (1,1) circle (2pt) node[above right] {$(1,1)$};
  \fill[black] (3,3) circle (2pt) node[above right] {$(3,3)$};
  \fill[black] (3,1) circle (2pt) node[above right] {$(3,1)$};

    \draw[green, thick] (1,1) -- (3,1) -- (3,3) ;
      \draw[<-, blue, thick] (2,1) arc (0:\angle:1) node[right, xshift=2mm, yshift=-1mm] {$\theta$};
       \draw (3,1) -- (3,1.2) -- (2.8,1.2) -- (2.8,1) -- cycle;
\end{tikzpicture}
\end{center}

Que pode ser calculado como $a = \frac{y2 -y1}{x2- x1}$(pode-se fazer isso algebricamente ou pensar geometricamente como sendo a tangente).

\subsection{Definição de "b"}

O parâemetro "b" é o ponto em que a reta encosta o eixo y, já que, quando $x = 0$, tem-se que $y = a \cdot 0 + b \xrightarrow[]{} y = b.$ Esse parâmetro, como dito anteriormente, indica se a reta está "para cima ou para baixo".
\section{Retas Paralelas e Perpendiculares}

Para saber se duas retas são paralelas ou perpendiculares, basta olhar o "a" da equação, pois ele quem dita a inclinação da reta. Para que duas retas sejam paralelas, basta que elas façam o mesmo ângulo com o semi-eixo x positivo. E elas não são paralelas caso contrário.

\begin{tikzpicture}
    % Draw the x and y axes
    \draw[->] (-2,0) -- (4,0) node[right] {$x$};
    \draw[->] (0,-2) -- (0,4) node[above] {$y$};
     \def\angle{45}
     \draw[<-, blue, thick] (1,0) arc (0:\angle:1) node[right, xshift=2mm, yshift=-1mm] {$\theta$};
     \draw[<-, blue, thick] (0,0) arc (0:\angle:1) node[right, xshift=2mm, yshift=-1mm] {$\theta$};
    % Draw the y = x line
    \draw[domain=-1.5:3.5,smooth,variable=\x,red] plot ({\x},{\x}) node[right] {$y = ax + b_1$};
    
    % Draw the y = x + 1 line
    \draw[domain=-1.5:3.5,smooth,variable=\x,blue] plot ({\x},{\x + 1}) node[right] {$y = ax + b_2$};
\end{tikzpicture}

Para que duas retas sejam paralelas, basta que o produto dos seus coeficientes angulares seja igual a \texbf{-1}. E elas não são paralelas caso contrário.
\begin{tikzpicture}
    \draw[->] (-2,0) -- (4,0) node[right] {$x$};
    \draw[->] (0,-2) -- (0,4) node[above] {$y$};
    
    \def\anglea{atan(2)}  % Calculate the angle of the green line
    \def\angleb{atan(-0.5)}  % Calculate the angle of the orange line

    % Draw the y = ax + b1 line
    \draw[domain=-1:2,smooth,variable=\x,black] plot ({\x},{2*\x}) node[right] {$y = a_1x + b_1$};

    % Draw the y = -1/a * x + b2 line
    \draw[domain=-1:2.5,smooth,variable=\x,black] plot ({\x},{-0.5*\x + 1}) node[right] {$y = a_2x + b_2$};  
 \coordinate (intersection) at (2/5,4/5);
    % Draw a square at the intersection point
    \fill[black] (intersection) circle (2pt);
    % Draw angle arc for y = ax + b1
    \draw[thick, blue] (0.5,0) arc (0:\anglea:0.5) node[right] {$\alpha$};

    % Draw angle arc for y = -1/a * x + b2
    \draw[thick, red] (2.5,0) arc (0:180+\angleb:0.5) node[above right] {$90^{\circ} + \alpha$}; 
\end{tikzpicture}

A prova disso se dá por transformações trigonométricas. Para isso, vamos desenhar a figura e perceber que a tangente do ângulo $\alpha$ é igual a "a". Dessa forma, temos que provar que $a_1 \cdot a_2 = -1$, ou, de forma equivalente, que : 

$$
\tan{\alpha} \cdot \tan{90^{\circ} + \alpha} = -1 
\iff
\frac{sen(\alpha)}{cos(\alpha} \cdot \frac{sen(90 + \alpha)}{cos(90 + \alpha)} = -1 
\iff
\frac{\cancel{sen(\alpha)}}{\cancel{cos(\alpha)}} \cdot \frac{\cancel{cos(\alpha)}}{\cancel{-sen(\alpha)}} = -1 
\\

$$

\section{Distância entre dois pontos}

Aprende-se na escola que a "menor distância entre dois pontos é uma  reta que liga eles". Agora aprender-se-a como calcular essa distância. Temos que a distância entre pontos $P_1 = (x_1,y_1)$ e $P_2 = (x_2, y_2)$ é dada por $d = \sqrt{(x_1-x_2)^2 + (y_1-y_2)^2}$, que se obtém através do teorema de Pitágoras.
\begin{center}
\begin{tikzpicture}
  % Draw x-axis
  \draw[thick, ->] (-1,0) -- (5,0) node[below] {$x$};
  
  % Draw y-axis
  \draw[thick, ->] (0,-1) -- (0,5) node[left] {$y$};
  
  % Plot points (1,2) and (4,3)
  \fill[black] (1,2) circle (2pt) node[above right] {$(x_1,y_1)$};
  \fill[black] (4,3) circle (2pt) node[above right] {$(x_2,y_2)$};
  \fill[black] (4,2) circle (2pt) node[above right] {};
  % Draw a red line connecting the points
  \draw[red, thick] (1,2) -- (4,3);

  \draw[blue, thick] (1,2) -- (4,2) -- (4,3);
    \fill[yellow] (4,2) -- (4,2.2) -- (3.8, 2.2) -- (3.8,2) -- cycle;
  
\end{tikzpicture}
\end{center}

Com o cateto horizontal valendo $x_2 - x_1$ e o vertical valendo $y_2 - y_1$, aplica-se teorema de Pitágoras e obtém-se a distância entre $P_1 $ e $P_2$.

\section{Equação do círculo}

A equação de um círculo de raio $R$ centrado em $(x_0, y_0)$ é dada por $$(x-x_0)^2 + (y-y_0)^2 = R^2$$.
\begin{center}
\begin{tikzpicture}
  % Draw x-axis
  \draw[thick, ->] (-1,0) -- (5,0) node[below] {$x$};
  
  % Draw y-axis
  \draw[thick, ->] (0,-3) -- (0,3) node[left] {$y$};
  
  % Draw a circle centered at (2,0) with a radius of 3
  \draw[blue,thick] (1,0) circle (2);
  \fill[thick] (1,0) circle(0.1) node[above right] {$(x_0, y_0)$};
 \fill[blue] (2,1.74) circle (2pt) node[above right] {$(x,y)$};
 \fill[blue] (2,0) circle (2pt);
 \draw[thick, green] (1,0) -- (2,0) -- (2,1.74) -- (1,0);
% \end{center}
\end{tikzpicture}
\end{center}

Isso se dá pois todos os pontos pertencentes ao círculo satisfazem a relação $(x - x_0)^2 + (y-y_0)^2 = R^2$ já que o Teorema de Pitágoras sempre vale e todos esses pontos são equidistantes de um centro $(x_0, y_0)$.

\section{Área de um triângulo}

A área de um triângulo de coordenadas $(x_a,y_a)$, $(x_b, y_b)$ e $(x_c,y_c)$ é dada por :
$$
\frac{1}{2} \cdot 
\begin{vmatrix}
x_a & y_a & 1 \\
  x_b & y_b & 1 \\
  x_c & y_c & 1 \\
\end{vmatrix}
= 
\frac{1}{2} \cdot |x_ay_b + x_c  y_a + x_b y_c - x_a y_c - y_a x_b - y_b x_c|
$$

Sendo esse o determinante da matriz acima. A prova disso pode ser encontrada em vários sites e usam coisas bem mais elementares do que Álgebra Linear.

\section{Vamos aprender com problemas ?}
\begin{tcolorbox}[colback=blue!5!white,colframe=blue!75!black,title=Problema 1]
(AMC 2004 10B) Um triângulo de lados 5,12 e 13 tem um círculo inscrito e circunscrito. Qual a distância entre os centros desses círculos ?
\end{tcolorbox}
\begin{center}
\begin{tikzpicture}[scale=0.5]
  % Define the vertices of the triangle
  \coordinate (A) at (0,0);
  \coordinate (B) at (5,0);
  \coordinate (C) at (0,12);
  
  % Draw the triangle
  \draw (A) -- (B) -- (C) -- cycle;
    \fill[black] (0,0) circle(3pt) node[above right] {$(0,0)$};
     \fill[black] (5,0) circle (3pt) node[above right] {$(5,0)$};
     \fill[black] (0,12) circle (3pt) node [above left] {$(0,12)$};
     \fill[black] (2,2) circle (3pt);
     \fill[black] (2.5, 6) circle (3pt);
     \draw[thick, green] (2,2) -- (2.5, 6);
  % Draw the inscribed circle
  \draw[thick, blue](2.5, 6) circle (6.5cm);
    \draw[thick, red](2, 2) circle (2cm);
     \draw[thick, ->] (-1,0) -- (8,0) node[below] {$x$};
     

  % Draw y-axis
  \draw[thick, ->] (0,-3) -- (0,14) node[left] {$y$};
  % Draw the circumscribed circle
  % \draw (barycentric cs:A=1,B=1,C=1) circle (5cm);
\end{tikzpicture}
\end{center}
Para começarmos a atacar esse problema, vamos usar o fato de que o triâgnulo 5,12,13 é retângulo com hipotenusa de valor 13. Dessa forma, é conveniente colocar o ponto de encontro dos catetos como sendo o ponto $(0,0)$. Agora, para calcular o valor do inraio, basta sabermos que a  área do triângulo é dada por $5 \cdot 12 / 2 = 30$. E que ela é $r \cdot s = [ABC]$, com $s$ sendo o semiperímetro do triânngulo. Dessa forma, tem-se que o inraio tem valor 2, dessa forma, o incentro é $(2,2)$. O circuncentro é dado por você unir duas mediatrizes, no caso, pegarei as dos catetos já que elas são paralelas aos eixos. Dessa forma o circuncentro é o ponto $(2.5, 6)$. Agora, basta usar a fórmula da distância para perceber que a distância é dada por : $\frac{\sqrt{65}}{2}$

\begin{tcolorbox}[colback=blue!5!white,colframe=blue!75!black,title=Problema 2]
(PUMAC 2012)Dois círculos centrados em O e P tem raios 5 e 6, respectivamente.
Círculo O passa pelo ponto P. Sejam os pontos de interseção dos círculos centrados em O e P como M e N.
Ache a área do triângulo MNP. \emoji{bomb}
\end{tcolorbox}
\begin{center}
\begin{tikzpicture}[scale  = 0.5]
  % Draw the x-axis
  \draw[thick, ->] (-7,0) -- (7,0) node[below] {$x$};
  
  % Draw the y-axis
  \draw[thick, ->] (0,-2) -- (0,12) node[left] {$y$};
  
  % Draw the circle centered at (0,0) with radius 5
  \draw[thick,red] (0,0) circle (5);
  \fill[black] (0,0) circle (0.1) node[above left] {$(0,0)$};
  % Draw the circle centered at (0,5) with radius 6
  \draw[thick,blue] (0,5) circle (6);
  \fill[black](0,5) circle(0.1) node [above left] {$(0,5)$};
  \fill[black](0,5) circle(0.1) node [above right] {P};
  \fill[red](-4.8,1.4) circle(0.1) node [above left] {M};
  \fill[red] (4.8, 1.4) circle(0.1) node [above left] {N};
  \draw[thick, green] (-4.8,1.4) -- (4.8,1.4) -- (0,5) -- cycle;
\end{tikzpicture}
\end{center}

Para resdolver esse problemas, é conveniente colocar os círcuos em posições fáceis de trabalhar. Então, seja O = $(0,0)$ e P = $(0,5)$. Dessa forma, temos que a equação do círculo O e P são, respectivamente : 
$$
x^2+ y^2 = 25 (1)
$$
e
$$
x^2 + (y-5)^2 = 36 = x^2 + y^2 - 10y + 25 = 36 (2)
$$

Com isso, temos que, substituindo a equação 1 em 2, que $-10y = 36-50 = -14 \xrightarrow{} y = 1.4$ Substituindo isso na equação 1, tem-se que $x^2 + 1.96 = 25 \xrightarrow{} x^2 = 23.04 \xrightarrow{} x = \pm \sqrt{\frac{2304}{100}} = \pm \frac{48}{10} = \pm 4.8$. E, usando que $base \cdot altura $ sobre 2, tem-se que a área do triângulo é :
$$9.6 \cdot 3.6 / 2 = 17,28.$$

\section{Problemas olímpicos \emoji{bomb}}

\begin{enumerate}
    \item (IMO) No quadrilátero convexo ABCD, as diagonais AC e BD são perpendiculares e os  lados opostos AB e CD não são paralelos.Sabemos que o ponto P, onde se intersectam as mediatrizes de AB e CD, está no interior de ABCD. Prove que ABCD é um quadrilátero cíclico se. e somente se, os triângulos ABP e CDP têm áreas iguais.
    \item 
    (2019 AIME II Problems/Problem 1)Dois pontos diferentes, $C$ e $D$, estãpo no mesmo lado da reta AB$AB$ tal que $\triangle ABC$ e $\triangle BAD$ são congruentes, comm $AB = 9$, $BC=AD=10$, and $CA=DB=17$.A interseção dessad duas regiões triangulares tem área $\frac{m}{n}$, em que m e n são inteiros primos entre si. Ache m + n.
    \item
    (Balcânica Adaptado) Seja ABC um triângulo acutângulo e M, N, P as projeções ortogonais  do baricêntrico de ABC sobre seus lados. Prove que : 
    $$
    \frac{2}{9} < \frac{[MNP]}{[ABC]} \leq \frac{1}{4}
    $$
    Denote [XYZ] como a área do triângulo XYZ.
    \item 
    (OMMC 2022 P11)Seja$ABC$ um triângulo tal que $AB = 7$, $BC = 8$, and $CA = 9$. Existe um único ponto $X$ tal que $XB = XC$ e $XA$ é tangente ao circuncírculo de $ABC$. Se $XA = \tfrac ab$, onde $a$ e $b$ são inteiros primos entre si, ache $a + b$.
    \begin{tcolorbox}[colback=red!5!white,colframe=red!75!black,title=PERIGO!!!!\emoji{warning}]
    \item 
    (ELMO 2020 P4 \emoji{bomb} \emoji{bomb})Seja o triângulo acutângulo escaleno ABC com ortocentro H e altura AD com D no lado BC. Seja M o ponto médio de BC e seja $D'$ a reflexão de D sobre M. Seja P o ponto na reta $D'H$ tal que as retas AP e BC sejam paralelas e sejam os circuncírculos de $\triangle AHP$ e $\triangle BHC$ se encontram novamente em $G \neq H$. Prove que $\angle MHG = 90^{\circ}$.
    \end{tcolorbox}
\end{enumerate}


\begin{thebibliography}{9}
\bibitem{texbook}
The Art of Coordinate Bashing, Beckman Math Club. Acessível através do link : \href{https://www.mit.edu/~ehjin/files/CoordBash.pdf}{https://www.mit.edu/~ehjin/files/CoordBash.pdf}.
\bibitem{texbook}
Shine, Carlos Yuzo. 21 aulas de Matemática Olímpica.
\bibitem{texbook}
ELMO 2020. Acessível através do link:
\href{https://web.evanchen.cc/exams/ELMO-2020.pdf}{https://web.evanchen.cc/exams/ELMO-2020.pdf}
\bibitem{texbook}
OMMC 2022 P11, achado na "thread" da Aops : \href{https://artofproblemsolving.com/community/q3h2858327p25365132}{https://artofproblemsolving.com/community/q3h2858327p25365132}
\end{thebibliography}




\end{document}
