\documentclass{article}
\usepackage{graphicx} % Required for inserting images
\usepackage{draftwatermark}
\usepackage{tikz}
\usepackage{amsfonts} 
\usepackage{amsmath}
\usepackage{emoji}
\SetWatermarkText{BASH}
\SetWatermarkScale{1}
\usepackage{fontspec}
\usetikzlibrary{calc}
\usepackage{tkz-euclide}
\setmainfont{Arial}
\usepackage{asymptote}
\usepackage{lipsum}
\usetikzlibrary{angles,quotes}
% \usepackage{hyperref}
% \tcbuselibrary{skins}
\usepackage{emoji}
\usetikzlibrary{decorations.markings}
\renewcommand*\contentsname{Summary}

\usepackage[most]{tcolorbox}
\renewcommand*\contentsname{Summary}
\usepackage{hyperref}
\title{Complex bash}
\author{Master Humberto}
\date{Copia não comédia}

\begin{document}

\maketitle
\tableofcontents

\section{Introdução aos complexos}

Os números complexos foram "inventados" para satisfazer aquelas equações sem solução, por exemplo, equações do segundo grau em que o delta fosse menor que zero. Usa-se a unidade imaginária \textit{i}, que é igual a $\sqrt{-1}$ multiplicada a um número real para se fazer a "parte imaginária do número complexo". Os números complexos são feitos da forma $z = a + bi$, em que $a,b \in \mathbb{R}$. Mas como usar isso em  geometria ? Existe um plano chamado de plano de Argand-Gauss em que o eixo x é substituido pela parte real do número complexo e o eixo y é substituído pela parte imaginária  do número complexo. Sendo z = $a + bi$, tem-se representado esse número como na figura abaixo.
\begin{center}
\begin{tikzpicture}
  % Draw the axes
  \draw[->] (-1,0) -- (4,0) node[right] {$\mathbb{R}$};
  \draw[->] (0,-1) -- (0,4) node[above] {$\mathbb{C}$};
  
  % Draw point "a" at (2,0)
  \fill (2,0) circle (2pt) node[below] {$a$};
  
  % Draw point "b" at (0,3)
  \fill (0,3) circle (2pt) node[left] {$b$};
  \draw[blue, thick] (2,0) -- (2,3) -- (0,3);
   \fill[red] (2,3) circle (2pt) node [right] {$z = a +bi$};
\end{tikzpicture}
\end{center}

Os números completos também podem ser representados na sua forma polar, que, ao invés de  ter um "a" e "b" como parâmetros, tem-se r e $\theta$, com o número sendo escrito como $z = r \cdot e^{i \theta}$, sendo $e^{i \theta} = cos(\theta) + i sen(\theta)$(a prova disso pode ser feita analisando os polinômios de Taylor de $e^x, sen(x), cos(x)$, deixo como exercício para o leitor. Mas o que é "r" e $"\theta"$ ? O "r" é a magnitude do número complexo z, ou seja, a distância de z à origem do sistema de coordendas, ou seja, se $z = a + bi$, $r = \sqrt{a^2 + b^2}$ por Pitágoras. Já $"\theta"$ é o ângulo que o número complexo faz com o semi-eixo x positivo, medido em radianos. Para simplificar, vou fazer uma  figura abaixo.
\begin{center}
\begin{tikzpicture}
  % Draw the axes
  \draw[->] (-1,0) -- (4,0) node[right] {$\mathbb{R}$};
  \draw[->] (0,-1) -- (0,4) node[above] {$\mathbb{C}$};
  
  % Draw point "a" at (2,0)
  \fill (2,0) circle (2pt) node[below] {$a$};
  
  % Draw point "b" at (0,3)
  \fill (0,3) circle (2pt) node[left] {$b$};
  \fill (2,3) circle (2pt) node [right] {z = a + bi};
  % Calculate the argument of z = 2 + 3i
  \pgfmathanglebetweenpoints{\pgfpoint{0cm}{0cm}}{\pgfpoint{2cm}{3cm}}
  \let\alpha\pgfmathresult
  \draw[blue, thick] (2,0) -- (2,3) -- (0,3);
  \draw[blue, thick] (0,0) -- (2,3) node[midway, above] {$r$};
  % Draw an arc to represent the argument theta
  \draw (0.5,0) arc (0:\alpha:0.5cm);
  
  % Label the arc as "theta"
  \node at (0.7,0.3) {$\theta$};
\end{tikzpicture}
\end{center}

\subsection{Conjugado complexo}

O conjugado complexo será bastante explorado aqui no material. Basicamente, se temos um número complexo $z = a+ bi$, $\overline{z}$(conjugado)$= a - bi$. Isso representa, por exemplo, a reflexão de z sobre o eixo dos reais. Como representado na figura abaixo.
\begin{center}
\begin{tikzpicture}[scale = 0.7]
  % Draw the axes
  \draw[->] (-1,0) -- (4,0) node[right] {$\mathbb{R}$};
  \draw[->] (0,-1) -- (0,4) node[above] {$\mathbb{C}$};
  
  % Draw point "a" at (2,0)
  \fill (2,0) circle (2pt) node[below] {$a$};
  
  % Draw point "b" at (0,3)
  \fill (0,3) circle (2pt) node[left] {$b$};
  
  % Calculate the argument of z = 2 + 3i
  \pgfmathanglebetweenpoints{\pgfpoint{0cm}{0cm}}{\pgfpoint{2cm}{3cm}}
  \let\alpha\pgfmathresult
  
  % Draw complex number 2 - 3i
  \fill (2,-3) circle (2pt) node[right] {$z = a - bi$};
    \fill (2,3) circle (2pt) node [right] {$z = a + bi$};
    \draw[red, thick] (0,0) -- (2,3) node[midway,below] {$r$};
    
  % Draw a line from the origin to 2 - 3i
  \draw[red, thick] (0,0) -- (2,-3);
  
  % Label the angle as "theta"
  \node at (0.7,0.3) {$\theta$};
  
  % Label the distance from the origin as "r"
  \node[red] at (1.2,-1.5) {$r$};
  
  % Draw an arc to represent the argument theta of 2 - 3i
  \draw (0.5,0) arc (0:\alpha:0.5cm);
  \draw(0.28,-0.4) arc (-\alpha:0:0.5cm);
   \node at (0.7,-0.3) {$\theta$};
   \draw[thick] (2,-3) -- (2,3);
  \node[red] at (2,1.5) {=};
  \node[red] at (2,-1.5) {=};
\end{tikzpicture}
\end{center}

E, além de ser a reflexão, o conjugado complexo tem a prorpiedade de que $z \cdot \overline{z} = |z|^2$, com $|z|$ sendo o módulo de z ou  "r".

\section{Transformações geométricas com números complexos}

Os números complexos, por serem parecidos com vetores, tem propriedades interessantes, como, por exemplo, a de que, se colocarmos m * z, com m sendo um número real, temos um novo número complexo de tamanho "m vezes" o número complexo z, assim como na multiplicação de vetor po um escalar. Além disso, se multiplicarmos um número complexo $z = r \cdot e^{i \theta}$ por $w = r' \cdot e^{i \alpha}$, você tem um vetor de magnitude $r \cdot r'$ e argumento $e^{i(\theta + \alpha)}$, devido ao fato de que a multiplicação de potências de mesma base faz-se somar os expoentes. Dessa forma, temos que a rotação e translação, além da dilatação de um número complexo são bem simples. E, caso quisermos, podemos rotacionar e transladar toda a figura apenas multiplicando os vértices dela por um complexo w, criando uma figura semelhante à anterior.
\begin{center}
\begin{tikzpicture}
  % Draw the x and y axes
  \draw[->] (-4,0) -- (4,0) node[right] {$\mathbb{R}$};
  \draw[->] (0,-1) -- (0,4) node[above] {$\mathbb{C}$};
  
  % Draw the original square
  \draw[blue] (2,2) -- (2,3) -- (3,3) -- (3,2) -- cycle;
  
  % Calculate the coordinates of the rotated square
  \coordinate (A) at (2,2);
  \coordinate (B) at (3,3);
  \coordinate (C) at (2,3);
  \coordinate (D) at (3,2);
  
  \coordinate (A_rotated) at (-2,2);
  \coordinate (B_rotated) at (-2,3);
  \coordinate (C_rotated) at (-3,3);
  \coordinate (D_rotated) at (-3,2);
  
  % Draw the rotated square
  \draw[red] (A_rotated) -- (B_rotated) -- (C_rotated) -- (D_rotated) -- cycle;
\draw[thick] (0,0) -- (2,2);
\draw[thick] (0,0) -- (3,3);
\draw[thick] (0,0) -- (2,3);
\draw[thick] (0,0) -- (3,2);
\draw[thick] (0,0) -- (-2,2);
\draw[thick] (0,0) -- (-2,3);
\draw[thick] (0,0) -- (-3,3);
\draw[thick] (0,0) -- (-3,2);
\node[thick] at (0,-2) {Quadrado rotacionado em 90 graus no sentido anti-horário};
\end{tikzpicture}
\end{center}

\section{Algumas fórmulas e noções}

Uma das noções mais básicas de números complexos são algumas "fórmulas" que são derivadas imediatamente do fato de que números complexos são como vetores. Disso, tem-se que : 
\begin{enumerate}
    \item Baricentro G do triângulo de vértices a,b,c é $\frac{a+b+c}{3}$
    \item O ponto médio do segmento de vértices a,b e $\frac{a+b}{2}$
    \item O paralelogramo ABCD de números complexos a,b,c,d respeita a relação a + c = b + d, com a volta sendo válida(ou seja, da equação, você prova que é um paralelogramo certo quadrilátero).
\end{enumerate}
\subsection{Reflexão Complexa}
Outra fórmula que podemos enunciar é a fórmula da reflexão complexa, que podemos achá-la usando o fato de que o conjugado de z é a reflexão de z sobre o eixo real, dessa forma, basta "levarmos" a figura ao eixo real.

\begin{tikzpicture}[scale = 0.4]
  % Draw the real axis
  \draw[->] (-4,0) -- (4,0) node[right] {$\mathbb{R}$};
  
  % Draw the imaginary axis
  \draw[->] (0,-4) -- (0,4) node[above] {$\mathbb{C}$};
  
  % Complex numbers
  \coordinate (z1) at (1,2);
  \coordinate (z2) at (1,1);
  \coordinate (z3) at (3,3);
  \coordinate (z1_reflected) at (2,1);
  
  % Reflect z1 over the line containing z2 and z3

  
  % Draw complex numbers
  \fill (z1) circle (3pt) node[right] {$z$};
  \fill (z2) circle (3pt) node[right] {$a$};
  \fill (z3) circle (3pt) node[right] {$b$};
  \fill (z1_reflected) circle (3pt) node[right] {$w$};
  
  % Draw the line containing z2 and z3
  \draw[dashed] (z2) -- (z3);
  
  % Draw the reflection line (perpendicular to the line containing z2 and z3)
  \draw[dashed] (z1) -- (z1_reflected);
\end{tikzpicture}
\begin{tikzpicture}[scale = 0.4]
  % Draw the real axis
  \draw[->] (-4,0) -- (4,0) node[right] {$\mathbb{R}$};
  
  % Draw the imaginary axis
  \draw[->] (0,-4) -- (0,4) node[above] {$\mathbb{C}$};
  
  % Complex numbers
  \coordinate (z1) at (0,1);
  \coordinate (z2) at (0,0);
  \coordinate (z3) at (2,2);
  \coordinate (z1_reflected) at (1,0);
  
  % Reflect z1 over the line containing z2 and z3

  
  % Draw complex numbers
  \fill (z1) circle (3pt) node[right] {$z-a$};
  \fill (z2) circle (3pt) node[right] {$0$};
  \fill (z3) circle (3pt) node[above, left] {$b-a$};
  \fill (z1_reflected) circle (3pt) node[right] {$w-a$};
  
  % Draw the line containing z2 and z3
  \draw[dashed] (z2) -- (z3);
  
  % Draw the reflection line (perpendicular to the line containing z2 and z3)
  \draw[dashed] (z1) -- (z1_reflected);
\end{tikzpicture}

\begin{tikzpicture}[scale = 0.4]
  % Draw the real axis
  \draw[->] (-4,0) -- (4,0) node[right] {$\mathbb{R}$};
  
  % Draw the imaginary axis
  \draw[->] (0,-4) -- (0,4) node[above] {$\mathbb{C}$};
  
  % Complex numbers
  \coordinate (z1) at (1,1);
  \coordinate (z2) at (0,0);
  \coordinate (z3) at (1,0);
  \coordinate (z1_reflected) at (1,-1);
  
  % Reflect z1 over the line containing z2 and z3

  
  % Draw complex numbers
  \fill (z1) circle (3pt) node[right] {$z-a/b-a$};
  \fill (z2) circle (3pt) node[right] {$0$};
  \fill (z3) circle (3pt) node[right] {$1$};
  \fill (z1_reflected) circle (3pt) node[left] {$w-a/b-a$};
  
  % Draw the line containing z2 and z3
  \draw[dashed] (z2) -- (z3);
  
  % Draw the reflection line (perpendicular to the line containing z2 and z3)
  \draw[dashed] (z1) -- (z1_reflected);
\end{tikzpicture}

Para isso, como na figura acima, primeiro subtrai-se a em  todas as coordendas para levá-la à origem e depois as divide por b-a. Dessa forma, temos que $\overline{\frac{z-a}{b-a}} = \frac{w-a}{b-a}$. E, fazendo as continhas, obtém-se que : 


\begin{tcolorbox}[colback=blue!5!white,colframe=blue!75!black,title=Reflexão Complexa\emoji{nerd-face}]
A reflexão de z sobre a reta AB, de coordendas a,b é dada por : 

$$\frac{(a-b) \overline{z} + \overline{a}b - a \overline{b}}{\overline{a}- \overline{b}}$$

\end{tcolorbox}

Para calcular o pé da perpendicular complexa usando isso, basta usar o fato de que o ponto médio entre o ponto e a reflexão é esse ponto.

\section{Colinearidade e Perpendicularidade}

Para descobrir se duas retas são colineares, basta usar a fórmula do coeficiente angular complexo, que é dada por : 

\begin{tcolorbox}[colback=blue!5!white,colframe=blue!75!black,title=Coeficiente angular complexo\emoji{nerd-face}]
O coeficiente angular da reta AB é dada por : 
$$\frac{a-b}{\overline{a} - \overline{b}}$$

\end{tcolorbox}

Você pode provar isso usando o fato de que isso é equivalente ao coeficiente angular da reta, que é aprendido no módulo de Coordbash. Dessa forma, para que três pontos sejam colineares, basta que : 

$$\frac{z-a}{z-b} = \overline{\frac{z-a}{z-b}}$$
% \footnote{Se a vida te der limões, faça uma limonada e,caso te dê muitas interseções, chore calado.}
\\
AB é perpendicular a CD se e somente se (usando que os coeficientes angulares são opostos) :

$$\frac{a-b}{\overline{a} - \overline{b}} = - \frac{c-d}{\overline{c} - \overline{d}}$$

\begin{tcolorbox}[colback=blue!5!white,colframe=blue!75!black,title=Área complexa\emoji{nerd-face}]
A área do triângulo de vértices a,b,c é dada por : 
$$
\frac{i}{4} \cdot 
\begin{vmatrix}
a & \overline{a} & 1 \\
b & \overline{b} & 1 \\
c & \overline{c} & 1 \\
\end{vmatrix}
$$
\end{tcolorbox}

\section{Círculo Unitário}

O círculo unitário é uma das ferramentas mais poderosas na hora de se resolver problemas de geometria com complexos, pois, como $z \cdot \overline{z} = |z|^2$, se o ponto fica no círculo  unitário (centro no 0 e raio 1), o módulo dele é igual a $1$ e, dessa forma, tem-se que $\overline{z} = \frac{1}{z}$, o que facilita bastante as contas em muitos casos.
\begin{center}
\begin{tikzpicture}
  % Draw the unit circle
  \draw (0,0) circle [radius=2];
   \draw[->] (-3,0) -- (3,0) node[right] {$\mathbb{R}$};
  \draw[->] (0,-3) -- (0,3) node[above] {$\mathbb{C}$};
  % Define the coordinates of the points
  \coordinate (A) at ({sqrt(2)}, {-sqrt(2)});
  \coordinate (B) at ({-sqrt(2)}, {-sqrt(2)});
  \coordinate (C) at (-1,{sqrt(3)});
% \coordinate(D) at (-0.9380927988944993,-1.264845275045361);
% \coordinate(E) at (0.4309463489851838,0.10419387283432113);
% \coordinate(F) at (-1.305777598610757,-0.8971604753291035);
% \coordinate(H) at (-0.9380927988944993,-0.944098960399264)
   \draw (A) -- (B) -- (C) -- cycle;
  % Draw and label the points
  \fill[red] (A) circle (2pt) node[below right] {$c$};
  \fill[red] (B) circle (2pt) node[below left] {$b$};
  \fill[red] (C) circle (2pt) node[above] {$a$};
  \fill[blue,thick] (0,0) circle (2pt) node[above right] {$O$};
\end{tikzpicture}
\end{center}
\begin{tcolorbox}[colback=blue!5!white,colframe=blue!75!black,title=Pé complexo(Círculo Unitário)\emoji{nerd-face}]
O pé completo do ponto z a uma corda AB do círculo unitário é dada por : 

$$\frac{1}{2} \cdot (z + a + b - ab \overline{z})$$

\end{tcolorbox}

A prova disso se dá usando o teorema do pé complexo com os pontos a e b no círculo unitário, ficando como exercício para o leitor provar.

\begin{tcolorbox}[colback=blue!5!white,colframe=blue!75!black,title=Ortocentro\emoji{nerd-face}]
    O ortocentro H com a,b,c estando no círculo unitário é dado por : 
    $$ H = a + b + c$$
    Em particular, se o é o circuncentro do $\triangle ABC$, a fórmula é dada por : 
    $$ H = a + b + c - 2o$$
\end{tcolorbox}

\subsection{Círculo dos nove pontos}

O círculo dos nove pontos tem o centro que é o ponto médio entre o circuncentro e o ortocentro de um triângulo é dado por : 
\begin{tcolorbox}[colback=blue!5!white,colframe=blue!75!black,title=Círculo dos nove pontos(círculo 
 unitário)\emoji{nerd-face}]
$$n_9 = \frac{a+b+c- o}{2}$$
\end{tcolorbox}
Esse círculo contém os pontos médios dos lados do triângulo, seus pés das alturas e os pontos médios dos segmentos que ligam os vértices ao ortocentro do triângulo. 
\\
Prova:
\begin{center}
\begin{asy}
/* Geogebra to Asymptote conversion, documentation at artofproblemsolving.com/Wiki go to User:Azjps/geogebra */
import graph;
size(6cm);
real labelscalefactor = 0.5; /* changes label-to-point distance */
pen dps = linewidth(0.7) + fontsize(10);
defaultpen(dps);
pen dotstyle = black;
real xmin = -4.6775254428172905, xmax = 5.774110566138461, ymin = -3.7721426858764358, ymax = 2.7679529005659314;
pen wrwrwr = rgb(0.3803921568627451,0.3803921568627451,0.3803921568627451);

/* draw figures */
draw(circle((0,0), 2), linewidth(2) + wrwrwr);
draw((-1,1.7320508075688772)--(-1.4142135623730951,-1.4142135623730951), linewidth(2) + wrwrwr);
draw((-1,1.7320508075688772)--(1.4142135623730951,-1.4142135623730951), linewidth(2) + wrwrwr);
draw((-1.4142135623730951,-1.4142135623730951)--(1.4142135623730951,-1.4142135623730951), linewidth(2) + wrwrwr);
draw((-1,1.7320508075688772)--(-1,-1.4142135623730951), linewidth(2) + wrwrwr);
draw((-1.4142135623730951,-1.4142135623730951)--(0.3660254037844388,-0.04818815858865656), linewidth(2) + wrwrwr);
draw((1.4142135623730951,-1.4142135623730951)--(-1.3660254037844388,-1.0481881585886563), linewidth(2) + wrwrwr);
draw(circle((-0.5,-0.5481881585886564), 1), linewidth(2) + wrwrwr);

/* dots and labels */
dot((-1,1.7320508075688772),dotstyle);
label("$A$", (-0.9693851222746063,1.8135370374726671), NE * labelscalefactor);
dot((-1.4142135623730951,-1.4142135623730951),dotstyle);
label("$B$", (-1.3840084070610246,-1.3391645430731152), NE * labelscalefactor);
dot((1.4142135623730951,-1.4142135623730951),dotstyle);
label("$C$", (1.4479468588386626,-1.3391645430731152), NE * labelscalefactor);
dot((-1,-1.4142135623730951),linewidth(4pt) + dotstyle);
label("$D$", (-0.9693851222746063,-1.3548107047631686), NE * labelscalefactor);
dot((0.3660254037844388,-0.04818815858865656),linewidth(4pt) + dotstyle);
label("$E$", (0.39965402560507673,0.014228443116513466), NE * labelscalefactor);
dot((-1.3660254037844388,-1.0481881585886563),linewidth(4pt) + dotstyle);
label("$F$", (-1.337069921990864,-0.9871259050469111), NE * labelscalefactor);
dot((-1,-1.0963763171773129),linewidth(4pt) + dotstyle);
label("$H$", (-0.9693851222746063,-1.0340643901170716), NE * labelscalefactor);
dot((0,0),linewidth(4pt) + dotstyle);
label("$O$", (0.031969225888819036,0.06116692818667399), NE * labelscalefactor);
dot((-0.5,-0.5481881585886565),linewidth(4pt) + dotstyle);
label("$n_{9}$", (-0.46870794819289363,-0.48644873096519886), NE * labelscalefactor);
clip((xmin,ymin)--(xmin,ymax)--(xmax,ymax)--(xmax,ymin)--cycle);
/* end of picture */
\end{asy}
\end{center}

Primerio : que o centro do círculo dos nove pontos é o circuncentro do triângulo formado pelos pés das alturas.

$$|\frac{a+b+c}{2} - \frac{a+b+c - bc/a}{2}| = |-bc/2a| = 1/2$$

Como o problema é simétrico, pode-se dizer que está provado.

Segundo : que o centro do círculo dos nove pontos passa pelos pontos médios dos lados do triângulo.

$$|\frac{a+b+c}{2} - \frac{b+c}{2}| = |a/2| = 1/2$$

Terceiro : que o  centro do círculo dos nove  pontos passa pelos pontos médios dos segmetos que ligam os vértices ao ortocentro.
\\
Dever de casa.
\\
\section{Mais fórmulas}
\begin{tcolorbox}[colback=blue!5!white,colframe=blue!75!black,title=Fórmula do quadrilátero cíclico\emoji{nerd-face}]

Os vértices ABCD de um quadrilátero estão numa mesma circunferência se e somente se:

$$\frac{b-a}{c-a} \div \frac{b-d}{c-d} \in \mathbb{R}$$

 \end{tcolorbox}

Pode-se provar isso usando de o fato de que queremos provar que os ângulos $\angle BAC = \angle BDC$, como na figura abaixo.
\begin{center}
\begin{tikzpicture}[scale = 2]
  % Draw the unit circle
  \draw (0,0) circle [radius=1];
  
  % Define the coordinates of the points
  \coordinate (A) at ({sqrt(2)/2}, {-sqrt(2)/2});
  \coordinate (B) at ({-sqrt(2)/2}, {-sqrt(2)/2});
  \coordinate (C) at (-1/2,{sqrt(3)/2});
  \coordinate (D) at (1,0);
  
  % Draw and label the points
  \fill (A) circle (1pt) node[below right] {$c$};
  \fill (B) circle (1pt) node[below left] {$b$};
  \fill (C) circle (1pt) node[above] {$a$};
  \fill (D) circle (1pt) node[right] {$d$};
  
  % Draw the angle arcs
  \draw[red] (A) -- (C);
  \draw[red] (B) -- (D);
  % \draw[red,->] (0.3,0) arc (0:60:0.3);
  % \draw[red,->] (-0.3,0) arc (180:240:0.3);
  
  % Draw the triangle
  \draw[thick] (A) -- (B) -- (C) -- (D) -- cycle;
  
  % Add angles BCA and BDA
  \pic [fill, blue, --, angle radius=0.3cm] {angle = B--C--A};
  \pic [fill, blue, --, angle radius=0.3cm] {angle = B--D--A};
  
\end{tikzpicture}
\end{center}

Para provar que os ângulos são iguais, pode-se usar argumentos e, caso esses argumentos sejam iguais, na hora da divisão o argumento final do número complexo será zero, ou seja, número real.

\begin{tikzpicture}[scale=1]
  % Draw x and y axis
  \draw[->] (-1,0) -- (3,0) node[right] {$\mathbb{R}$};
  \draw[->] (0,-1) -- (0,3) node[above] {$\mathbb{C}$};
  
  % Define the coordinates of the points
  \coordinate (A) at (2,1);
  \coordinate (B) at (1,1);
  \coordinate (C) at (2,2);
  
  % Draw and label the points
  \fill (A) circle (1pt) node[above right] {$b$};
  \fill (B) circle (1pt) node[below left] {$a$};
  \fill (C) circle (1pt) node[below right] {$c$};
  
  % Draw the angle between the points
  \pic [draw, red, --, angle eccentricity=1.3, angle radius=0.5cm, "$\theta$"] {angle = A--B--C};
  \draw[thick,blue] (2,1) -- (1,1);
  \draw[thick,blue] (1,1) -- (2,2);
\end{tikzpicture}

\begin{tikzpicture}[scale=1]
  % Draw x and y axis
  \draw[->] (-1,0) -- (3,0) node[right] {$\mathbb{R}$};
  \draw[->] (0,-1) -- (0,3) node[above] {$\mathbb{C}$};
  
  % Define the coordinates of the points
  \coordinate (A) at (1,0);
  \coordinate (B) at (0,0);
  \coordinate (C) at (1,1);
  
  % Draw and label the points
  \fill (A) circle (1pt) node[above right] {$b-a$};
  \fill (B) circle (1pt) node[below left] {$a-a$};
  \fill (C) circle (1pt) node[below right] {$c-a$};
  
  % Draw the angle between the points
  \pic [draw, red, --, angle eccentricity=1.3, angle radius=0.5cm, "$\theta$"] {angle = A--B--C};
  \draw[thick,blue] (B) -- (A);
  \draw[thick,blue] (B) -- (C);
\end{tikzpicture}

\begin{tikzpicture}[scale=1]
  % Draw x and y axis
  \draw[->] (-1,0) -- (3,0) node[right] {$\mathbb{R}$};
  \draw[->] (0,-1) -- (0,3) node[above] {$\mathbb{C}$};
  
  % Define the coordinates of the points
  \coordinate (A) at (1,0);
  \coordinate (B) at (0,0);
  \coordinate (C) at (1/2,1);
  
  % Draw and label the points
  \fill (A) circle (1pt) node[above right] {$1$};
  \fill (B) circle (1pt) node[below left] {$0$};
  \fill (C) circle (1pt) node[below right] {$c-a/b-a$};
  
  % Draw the angle between the points
  \pic [draw, red, --, angle eccentricity=1.3, angle radius=0.5cm, "$\theta$"] {angle = A--B--C};
  \draw[thick,blue] (B) -- (A);
  \draw[thick,blue] (B) -- (C);
\end{tikzpicture}

Então, com isso, obtemos que o número complexo $\frac{c-a}{b-a} = r \cdot e^{i \theta}$, sendo $\theta$ igual a $\angle BAC$. Fazendo o mesmo processo com $\angle DAC$, obtemos $r' \cdot (e^{i\theta}$. Dessa forma, ao se dividir um pelo outro, sobra-se $\frac{r}{r'}$, que é real.



\begin{tcolorbox}[colback=blue!5!white,colframe=blue!75!black,title=Triângulos semelhantes complexos\emoji{nerd-face}]

Dois triângulos ABC e XYZ são semelhantes(nessa orientação) se e somente se :

$$\frac{c-a}{b-a} = \frac{z-x}{y-x}$$

\end{tcolorbox}

Isso pode ser provado provando que :

$$\frac{AC}{AB} = \frac{XZ}{XY} \iff |\frac{c-a}{b-a}| = |\frac{z-x}{y-x}| \iff \angle BAC = \angle ZXY$$, implicando no resultado desejado.

% \section{Mais fórmulas}

\begin{tcolorbox}[colback=blue!5!white,colframe=blue!75!black,title=Centro de roto-homotetia complexo\emoji{nerd-face}]

O centro de roto-homotetia P que leva XY em ZW é dado por : 

$$p = \frac{xw - yz}{x+w-y-z}$$

\end{tcolorbox}

Prova : Aplique a fórmula da semelhança de triângulos e abre a conta(vai por mim, pode abrir a conta sem medo).

\begin{center}
\definecolor{rvwvcq}{rgb}{0.08235294117647059,0.396078431372549,0.7529411764705882}
\definecolor{wrwrwr}{rgb}{0.3803921568627451,0.3803921568627451,0.3803921568627451}
\begin{tikzpicture}[line cap=round,line join=round,>=triangle 45,x=1cm,y=1cm]
\clip(-0.11308219939814423,-0.6076674595565992) rectangle (7.191539965405895,3.9631889250064143);
\draw [line width=2pt,color=wrwrwr] (2.2459011050313906,1.2268059740956627) circle (1.1606931867177241cm);
\draw [line width=2pt,color=wrwrwr] (4,2) circle (1.5492860591462407cm);
\draw [line width=2pt,color=wrwrwr] (1.3310032945103445,0.5125436483380026)-- (3.0220816079192514,3.201650078639824);
\draw [line width=2pt,color=wrwrwr] (1.1286869722416735,1.5415141805153014)-- (4.389796993819262,3.499448430815262);
\draw [line width=2pt,color=wrwrwr] (1.1286869722416735,1.5415141805153014)-- (1.3310032945103445,0.5125436483380026);
\draw [line width=2pt,color=wrwrwr] (3.0220816079192514,3.201650078639824)-- (4.389796993819262,3.499448430815262);
\begin{scriptsize}
\draw [fill=rvwvcq] (1.3310032945103445,0.5125436483380026) circle (2.5pt);
\draw[color=rvwvcq] (1.1740863850529776,0.6307284388925903) node {$Y$};
\draw [fill=wrwrwr] (2.493316819139523,2.3608227897140672) circle (2pt);
\draw[color=wrwrwr] (2.2386706074356575,2.467819043214567) node {$P$};
\draw [fill=rvwvcq] (1.1286869722416735,1.5415141805153014) circle (2.5pt);
\draw[color=rvwvcq] (1.0717877173151412,1.6586243722632201) node {$X$};
\draw [fill=wrwrwr] (3.0220816079192514,3.201650078639824) circle (2pt);
\draw[color=wrwrwr] (2.863553637241936,3.3098189035288064) node {$Z$};
\draw [fill=wrwrwr] (4.389796993819262,3.499448430815262) circle (2pt);
\draw[color=wrwrwr] (4.235904058922934,3.605065607794838) node {$W$};
\end{scriptsize}
\end{tikzpicture}
\end{center}

\begin{tcolorbox}[colback=blue!5!white,colframe=blue!75!black,title=Circuncentro complexo\emoji{nerd-face}]

O centro complexo X de um triângulo ABC é dado por : 

$$
x = 
\frac{
\begin{vmatrix}
    a & a \overline{a} & 1 \\
    b & b \overline{b} & 1 \\
    c & c \overline{c} & 1 \\
\end{vmatrix}
}
{
\begin{vmatrix}
a & \overline{a} & 1 \\
b & \overline{b} & 1 \\
c & \overline{c} & 1
\end{vmatrix}
}
$$

\end{tcolorbox}
A prova desse teorema se dá por pegar um círculo de raio r equidistante a a,b,c, fazer um sistema de equações e o centro será a solução do sistema de equações por regra de Cramer.
Em particular, se c = 0 (circunferência que passa pelo centro, tem-se que : 

$$x = \frac{ab(\overline{a} - \overline{b})}{\overline{a} b - a \overline{b}}$$

\begin{tcolorbox}[colback=blue!5!white,colframe=blue!75!black,title=Equação da corda(círculo unitário)\emoji{nerd-face}]
Aplicando a fórmula da colinearidade para um ponto z sob a conta AB, tem-se que : 

$$z + ab \overline{z} = a + b$$
\end{tcolorbox}

Em particular, se colocarmos as tangentes por a e b(substituindo a=a e b=a), tem-se que o encontro das tangentes saindo do círculo unitário é dada por : 

$$p = \frac{2ab}{a+b}$$

\section{Incentro}

Quando se tem o incentro na figura, o $A = a^2$, $B = b^2$ e $C = C^2$, isso se deve pois as fórmulas do incentro usam inversão raíz de bc(um material de inversão com raíz de bc está disponível na internet através desse \href{https://www.obm.org.br/content/uploads/2017/01/A_inversao_que_nao_muda_ABC_Carlos_Shine-1.pdf}{link}) e, para se calcular e tirar a raiz quadrada de um número complexo não funciona muito
bem pois começam a ocorrer alguns problemas. Dessa forma, se chama os números complexos dessa forma quando se há um incentro na figura.
\begin{center}
\begin{tikzpicture}
  % Draw the unit circle
  \draw (0,0) circle [radius=2];
   \draw[->] (-3,0) -- (3,0) node[right] {$\mathbb{R}$};
  \draw[->] (0,-3) -- (0,3) node[above] {$\mathbb{C}$};
  % Define the coordinates of the points
  \coordinate (A) at ({sqrt(2)}, {-sqrt(2)});
  \coordinate (B) at ({-sqrt(2)}, {-sqrt(2)});
  \coordinate (C) at (-1,{sqrt(3)});
  \coordinate (I) at (-0.4258640783629533,-0.40850494614637717);
  \coordinate (D) at (0,-2);
  \coordinate (E) at (1.4619496720014715,1.3933517057112954);
  \coordinate (F) at (-1.95,0.436684321789468);
  \coordinate (X) at (0,2);
  \coordinate (Y) at (-1.461, -1.39);
  \coordinate (Z) at (1.95, -0.43);
% \coordinate(D) at (-0.9380927988944993,-1.264845275045361);
% \coordinate(E) at (0.4309463489851838,0.10419387283432113);
% \coordinate(F) at (-1.305777598610757,-0.8971604753291035);
% \coordinate(H) at (-0.9380927988944993,-0.944098960399264)
   \draw (A) -- (B) -- (C) -- cycle;
  % Draw and label the points
  \fill[red] (A) circle (2pt) node[below right] {$c$};
  \fill[red] (B) circle (2pt) node[below left] {$b$};
  \fill[red] (C) circle (2pt) node[above] {$a$};
  % \fill[purple] (I) circle (2pt) node[above] {$I$};
  \fill[blue,thick] (0,0) circle (2pt) node[above right] {$O$};
  \draw[thick] (C) -- (D);
  \draw[thick] (B) -- (E);
  \draw[thick] (A) -- (F);
  \fill[purple] (D) circle (2pt) node[below right] {$d$} ;
  \fill[purple] (E) circle (2pt) node[right] {$e$};
  \fill[purple] (F) circle (2pt) node[right] {$f$};
   \fill[purple] (I) circle (2pt) node[above] {$I$};
\fill[cyan] (X) circle (2pt) node[above] {$X$};
\fill[cyan] (Y) circle (2pt) node[above] {$Y$};
\fill[cyan] (Z) circle (2pt) node[above] {$Z$};
   
\end{tikzpicture}
\end{center}

Tem-se que os pontos $D = -bc, E = -ac, F = -ab$, com D,E,F sendo os pontos médios dos arcos menores BC, AC e AB do triângulo ABC (com A,B,C no círculo unitário). Em particular, caso você queira o ponto médio do arco maior, basta pegar a antípoda, obtendo $X = bc$, $Y = ac$ e $Z = ab$.

\begin{tcolorbox}[colback=blue!5!white,colframe=blue!75!black,title=Incentro (círculo unitário)\emoji{nerd-face}]
O incentro I do triângulo ABC, com os vértices no círculo unitário, é dado por : 
$$I = -ab -ac  -bc$$
\end{tcolorbox}

Os ex-incentros relativos aos vértices A,B,C são dados por : 

$$I_a = -bc + ac + ab$$
$$I_b = -ac + bc + ab$$
$$I_c = -ab + ac + bc$$
(isso pode ser provado intersectando duas retas que saem de um vértice é são perpendiculares à bissetriz interna do ângulo).

\section{Vamos aprender com problemas ?}

\begin{tcolorbox}[colback=purple!5!white,colframe=purple!75!black,title=(OBM 2015 P1 N3)\emoji{nerd-face}]
Seja ABC um triângulo escaleno e acutângulo e N o centro do círculo que passa pelos pés das três alturas do
triângulo. Seja D a interseção das retas tangentes ao circuncírculo de ABC e que passam por B e C. Prove que
A, D e N são colineares se, e somente se, $\angle BAC = 45^{\circ}$

\end{tcolorbox}

\begin{center}
\definecolor{wrwrwr}{rgb}{0.3803921568627451,0.3803921568627451,0.3803921568627451}
\definecolor{rvwvcq}{rgb}{0.08235294117647059,0.396078431372549,0.7529411764705882}
\begin{tikzpicture}[line cap=round,line join=round,>=triangle 45,x=1cm,y=1cm]
 \draw[->] (-3,0) -- (3,0) node[right] {$\mathbb{R}$};
  \draw[->] (0,-3) -- (0,3) node[above] {$\mathbb{C}$};
\clip(-6.3577013991941405,-3.9471984270403873) rectangle (10.489125750721353,4.068032111245401);
\draw [shift={(-2.28,-0.96)},line width=1.6pt,fill=black,fill opacity=0.10000000149011612] (0,0) -- (0:0.3710754878836012) arc (0:45:0.3710754878836012) -- cycle;
\draw [line width=2pt,color=wrwrwr] (-2.28,-0.96)-- (2.28,-0.96);
\draw [line width=2pt,color=wrwrwr] (-2.28,-0.96)-- (0.944406922210657,2.264406922210656);
\draw [line width=2pt,color=wrwrwr] (0.944406922210657,2.264406922210656)-- (2.28,-0.96);
\draw [line width=2pt,color=wrwrwr] (0,-0.01559307778934355) circle (2.4678542166666575cm);
\draw [line width=2pt,color=wrwrwr] (0.944406922210657,0.3755930777893432)-- (0.944406922210657,2.264406922210656);
\draw [line width=2pt,color=wrwrwr] (2.28,-0.96)-- (0.944406922210657,0.3755930777893432);
\draw [line width=2pt,color=wrwrwr] (2.28,-0.96)-- (3.2244069222106546,1.32);
\draw [line width=2pt,color=wrwrwr] (3.2244069222106546,1.32)-- (0.7543301738598163,2.3341495514200816);
\draw [line width=2pt,color=wrwrwr] (-2.28,-0.96)-- (3.2244069222106546,1.32);
\begin{scriptsize}
\draw [fill=rvwvcq] (-2.28,-0.96) circle (2.5pt);
\draw[color=rvwvcq] (-2.226394300756714,-0.5394885299759818) node {$A$};
\draw [fill=rvwvcq] (2.28,-0.96) circle (2.5pt);
\draw[color=rvwvcq] (2.5975870417301015,-0.9105640178595833) node {$B$};
\draw [fill=rvwvcq] (0.944406922210657,2.264406922210656) circle (2.5pt);
\draw[color=rvwvcq] (1.0390699926189766,2.528068836528455) node {$C$};
\draw[color=black] (-1.9790106421676463,-0.7868721885650495) node {$45\textrm{\degre}$};
\draw [fill=wrwrwr] (0,-0.01559307778934355) circle (2pt);
\draw[color=wrwrwr] (-0.024679739314013384,0.2768775433679409) node {$O$};
\draw [fill=wrwrwr] (0.944406922210657,0.3755930777893432) circle (2pt);
\draw[color=wrwrwr] (1.0390699926189766,0.610845482463182) node {$H$};
\draw [fill=wrwrwr] (0.4722034611053285,0.18) circle (2pt);
\draw[color=wrwrwr] (0.5319334925113884,0.004755518919966645) node {$N$};
\draw [fill=wrwrwr] (3.2244069222106546,1.32) circle (2pt);
\draw[color=wrwrwr] (3.5994908590158246,1.7364411290434394) node {$D$};
\end{scriptsize}
\end{tikzpicture}

\end{center}

Para começar a atacar esse problema, podemos criar uma configuração tal que A reta AB seja paralela ao eixo real. Dessa forma, obtem-se que o número (a-b) é real e, assim, tem-se que ele é igual a seu conjugado.

Dessa forma, $a - b = \overline{a} - \overline{b} \iff -ab = 1$
E, usando o coeficiente ângulo da reta AC com AB, obtemos que :

$$\frac{a-c}{\overline{a}-\overline{c}} = -1 \iff ac = 1$$

Pela fórmula da colinearidade, queremos que o número complexo abaixo seja real : 

$$\frac{a - \frac{a+b+c}{2}}{a - \frac{2bc}{b+c}} = \frac{ab + ac -b^2 -c^2 - 2bc}{2ab + 2ac - 4bc}$$
Que tem conjugado igual a 
$$\frac{abc^2 + ab^ 2 c - a^2c^2 -a^2b^2 -2a^2bc}{2abc^2 + 2ab^2 - 4a^2bc}$$

Que não são iguais.

Mas, fazendo que $\angle BAC = 45^{\circ}$, obtemeos que : 

$$\frac{a - \frac{a+b+c}{2}}{a - \frac{2bc}{b+c}} = \frac{ab + ac -b^2 -c^2 - 2bc}{2ab + 2ac - 4bc} = \frac{-1+1-b^2 -c^2 -2bc}{-2+2 - 4bc}= \frac{b^2 + 2bc + c^2}{4bc}$$
Que possuí conjugado igual a : 
$$\frac{\frac{1}{b^2} + \frac{2}{bc} + \frac{1}{c^2}}{\frac{4}{bc}} = \frac{c^2 + 2bc + b^2}{4bc}$$. Que são iguais. Dessa forma, A,D,N são colineares se e somente se $\angle BAC = 45^{\circ}$.


\begin{tcolorbox}[colback=purple!5!white,colframe=purple!75!black,title=OBM N3 P2 2022\emoji{nerd-face}]
Seja ABC um triângulo acutângulo, com AB < AC. Sejam K o ponto médio do arco BC da
circunferência circunscrita a ABC que não contém A e P o ponto médio do lado BC. Os pontos $I_B$ e $I_C$.
são os exincentros relativos aos vértices B e C, respectivamente.
Seja Q a reflexão de K pelo ponto A. Mostre que P, Q, $I_B$ e $I_C$ estão sobre uma mesma circunferência.

\end{tcolorbox}
\begin{center}
\begin{tikzpicture}
  % Define the coordinates of the triangle vertices
  \coordinate (A) at ({sqrt(2)}, {-sqrt(2)});
  \coordinate (B) at ({-sqrt(2)}, {-sqrt(2)});
  \coordinate (C) at (-1, {sqrt(3)});
    \coordinate (K) at (0,-2);
    \coordinate (P) at (0, {-sqrt(2)});
    \coordinate (Q) at (-2, 5.46);
\coordinate (I_b) at (-3.57,1.04);
\coordinate (I_c) at (3.57, 2.97);
\coordinate (O) at (-0.08, 2.29);
  % Draw the triangle ABC
  \draw (A) -- (B) -- (C) -- cycle;

  % Calculate the coordinates of the ex-incenters relative to sides B and C
  % \coordinate (I_b) at (intersection of B--B+(C-B)*2 and A--C);
  % \coordinate (I_c) at (intersection of C--C+(B-C)*2 and A--B);
\draw[cyan] (0,0) circle [radius=2];
\draw[red] (O) circle [radius = 3.71];
  % Draw the ex-incenters I_b and I_c and label them
  \fill (A) circle (2pt)  node[below] {$C$};
  \fill (B) circle (2pt) node[below] {$B$};
  \fill (C) circle (2pt) node[above right] {$A$};
  \fill (K) circle (2pt)  node[below] {$K$};
  \fill (P)  circle (2pt) node[above] {$P$};
  \fill (Q) circle (2pt) node[below] {$Q$};
  \fill (I_b) circle (2pt) node[above right] {$I_c$};
  \fill (I_c) circle (2pt) node[above right] {$I_b$};
  \draw[thick] (I_c) -- (Q) -- (P);
  \draw[thick] (P) -- (I_b) -- (I_c);
\end{tikzpicture}
\end{center}

Para resolver esse problema, precisamos calcular os pontos $P, I_b, I_c, K$ e colocá-los na fórmula do quadrilátero cíclico. $P = \frac{b^2+c^2}{2}$, $I_b = -ac + ab +  bc$ e $I_c = -ab + ac + bc$. O ponto Q será :
$$Q = 2A - K \iff Q = 2a^2 + bc$$.

Agora, vamos jogar na fórmula da colinearidade (simples assim, \emoji{pile-of-poo}).

$$\frac{\frac{b^2 + c^2}{2} + ab - ac - bc}{-ac + ab + bc + ab - ac - bc} \div \frac{\frac{b^2 + c^2}{2} - 2a^2 - bc}{-ac + ab + bc - 2a^2 - bc} \in \mathbb{R}$$
$\iff$
$$\frac{b^2 + c^2 + 2ab - 2ac - 2bc}{-4ac + 4ab} \div 
\frac{b^2 + c^2 - 4a^2 - 2bc}{-2ac + 2ab - 4a^2}
$$
$$\iff$$
$$
\frac{(b-c)^2 + 2a(b-c)}{4a(b-c)} \div \frac{(b - c + 2a)(b-c - 2a)}{2a(-c+b-2a)}
$$
$$\iff$$
$$
(\frac{b-c}{4a} + 1/2) \div (\frac{c-b + 2a}{2a}) = (\frac{b-c}{4a} + 1/2) \cdot (\frac{2a}{-c+b+2a})
$$
$$\iff$$
$$
\frac{b-c +2a}{4a} \cdot \frac{2a}{-c+b-2a} = \frac{1}{2}
$$

Que é um número real. Logo, os quatro pontos são cíclicos.

\section{Problemas Olímpicos \emoji{bomb}}

\begin{enumerate}
    \item (EGMO 2023 P2)
 É dado um triângulo acutângulo $ABC$. Seja $D$ o ponto no seu circuncírculo tal que $AD$ é um diâmetro. Suponha que os pontos $K$ e $L$ estão nos segmentos $AB$ e $AC$, respectivamente, e que $DK$ e $DL$ são tangentes a $AKL$.
Mostre que a reta $KL$ passa pelo ortocentro do triângulo $ABC$.
    \textit{Convém usar o círculo de AKL como círculo unitário}
\item (OBM 2017 P5 N3)
No triângulo $ABC$, seja $r_A$ a reta que passa pelo ponto médio de $BC$ e é perpendicular à bissetriz interna do ângulo $\angle{BAC}$. Defina $r_B$ e $r_C$ similarmente. Seja $H$ e $I$ o ortocentro e incentro de $ABC$, respectivamente. Suponha que as três retas $r_A$, $r_B$, $r_C$ definam um triângulo.Prove que o circuncentro desse triângulo é o ponto médio de $HI$.
\item 
(IMO 2012) Dado um triângulo ABC, o ponto J é o centro da circunferência ex-inscrita oposta ao vértice A. Esta circunferência ex-inscrita é
tangente ao lado BC em M, e às retas AB e AC em K e L, respectivamente. As retas LM e BJ intersectam-se em F, e as retas KM e CJ
intersectam-se em G. Seja S o ponto de interseção das retas AF e BC, e
seja T o ponto de interseção das retas AG e BC.
Prove que M é o ponto médio de ST.
\item 
(OMCPLP/2019) Seja ABC um triângulo com $AC \neq BC$. No triângulo
ABC, sejam G seu baricentro (encontro das medianas), I seu incentro
(encontro das bissetrizes internas) e O seu circuncentro (centro da circunferência que passa pelos vértices). Prove que IG é paralelo a AB se,
e somente se, CI é perpendicular a IO
\item 
Dado um quadrilátero cíclico ABCD, as diagonais AC e BD se encontram em E e as retas
AD e BC se encontram em F. Os pontos médios de AB e CD são G e H, respectivamente. Mostre que EF
é tangente à circunferência que passa pelos pontos E, G e H.
\tcbset{colframe=red!75!black,fonttitle=\bfseries}
\begin{tcolorbox}[enhanced,title=ELMO SL 2013 G7 \emoji{fire},
interior style={left color=red!20!white,
right color=yellow!50!white}]
\item 
Seja $ABC$ um triângulo inscrito em um círculo $\omega$, e as medianas de $B$ e $C$ intersectam $\omega$ em $D$ e $E$ respectivamente. Seja $O_1$ o centro do círculo passando por $D$ tangente a $AC$ em $C$, e seja $O_2$ o centro do círculo por E $E$ tangente a $AB$ em $B$. Prove que $O_1$, $O_2$, e o centro do círculo de nove pontos de $ABC$ são colineares.
\end{tcolorbox}

\item 
(RMM 2019)
Seja $ABCD$ um trapézio isósceles com $AB\parallel CD$. E seja $E$ o ponto médio de $AC$. Denote por $\omega$ e $\Omega$ os circuncírculos dos triângulos $ABE$ e $CDE$, respectivamente. Seja $P$ a interseção da tangente por $\omega$ em $A$ com a tangente em $\Omega$ por $D$.Prove que $PE$ é tangente a $\Omega$.
\item 
(EGMO/2017)
 Seja ABC um triângulo acutângulo sem dois lados com o mesmo tamanho. A reflexão do
baricentro G e do circuncentro O do ABC pelos lados BC, CA, AB são G1, G2, G3 e O1, O2, O3, respectivamente. Mostre que os circunírculos dos triângulos G1G2C, G1G3B, G2G3A, O1O2C, O1O3B, O2O3A e
ABC têm um ponto em comum.
% \end{tcolorbox}
\end{enumerate}

% Que tem conjugado igual a :

% $$
% (\frac{c/bc-b/bc}{4/a} + 1/2) \cdot (\frac{2/a}{bc/abc - ab/abc + 2bc/abc})
% $$
% $$\iff$$
% $$
% \frac{}
% $$


% Que tem conjugado igual a :
% $$\frac{ac^2/ab^2c^2 + ab^2/ab^2c^2 + 2bc^2/ab^2c^2 - 2b^2c/ab^2c^2 - 2abc/ab^2c^2}{-4b/abc + 4c/abc} \div 
% $$
% $$
% \frac{a^2c^2/a^2b^2c^2 + a^2b^2/a^2b^2c^2 - 4b^2c^2/a^2b^2c^2 - 2a^2bc/a^2b^2c^2}{-2ab/a^2bc + 2ac/a^2bc - 4bc/a^2bc}
% $$
% $\iff$
% $$
% \frac{ac^2 + ab^2 + 2bc^2 - 2b^2c - 2abc}{-4ab + 4ac} \div 
% $$
% $$
% \frac{a^2c^2 + a^2b^2 - 4b^2c^2 - 2a^2bc}{-2ab^2c + 2abc^2 - 4b^2c^2}
% $$
% $$\iff$$


% $$\frac{a- \frac{2bc}{b+c}}{a - \frac{a+b+c}{2}}$$
% $$\iff$$
% $$\frac{\frac{-1 - 2bc}{b+c}}{\frac{a-b-c}{2}}$$
% Que tem conjugado igual a :
% $$\frac{\frac{-bc - 2}{c/bc + b/bc}}{\frac{bc/abc - ac/abc - ab/abc}{2}}$$
% $$\frac{\frac{-b^2c^2 - 2bc^2}{ab+ac}}{\frac{a-b-c}{2}} = \frac{\frac{-1 - 2bc}{b+c}}{\frac{a-b-c}{2}}$$
% $$\iff$$

% $$1 + 2bc = -bc(bc+c)^2$$
\begin{thebibliography}{9}
\bibitem{texbook}
Chen, Evan. Euclidean Geometry in Mathematical Olympiads.
\bibitem{texbook}
Paiva, Gabriel. Complex Bash. Disponível em : \href{https://www.obm.org.br/content/uploads/2021/11/Complex_Bash_Gabriel_Ribeiro_Paiva_SO2021.pdf}{https://www.obm.org.br/content/uploads/2021/11/Complex_Bash_Gabriel_Ribeiro_Paiva_SO2021.pdf}
\bibitem{texbook}
Andreescu, Titu. Lemmas in Olympiad Geometry. 
\bibitem{texbook}
Shine, Carlos Yuzo. A inversão $\sqrt{bc}$ que não muda bc. Disponível em :
\href{https://www.obm.org.br/content/uploads/2017/01/A_inversao_que_nao_muda_ABC_Carlos_Shine-1.pdf}{https://www.obm.org.br/content/uploads/2017/01/A_inversao_que_nao_muda_ABC_Carlos_Shine-1.pdf}

\bibitem{texbook}
Miyazaki, Rafael. Geometria com Complexos. Disponível em
\href{https://www.obm.org.br/content/uploads/2020/02/23_SO_Rafael_Miyazaki_Nivel_3_Complexos_compressed.pdf}{https://www.obm.org.br/content/uploads/2020/02/23_SO_Rafael_Miyazaki_Nivel_3_Complexos_compressed.pdf}
\end{thebibliography}
\end{document}
